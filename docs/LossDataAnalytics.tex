% Options for packages loaded elsewhere
\PassOptionsToPackage{unicode}{hyperref}
\PassOptionsToPackage{hyphens}{url}
%
\documentclass[
]{book}
\usepackage{lmodern}
\usepackage{amssymb,amsmath}
\usepackage{ifxetex,ifluatex}
\ifnum 0\ifxetex 1\fi\ifluatex 1\fi=0 % if pdftex
  \usepackage[T1]{fontenc}
  \usepackage[utf8]{inputenc}
  \usepackage{textcomp} % provide euro and other symbols
\else % if luatex or xetex
  \usepackage{unicode-math}
  \defaultfontfeatures{Scale=MatchLowercase}
  \defaultfontfeatures[\rmfamily]{Ligatures=TeX,Scale=1}
\fi
% Use upquote if available, for straight quotes in verbatim environments
\IfFileExists{upquote.sty}{\usepackage{upquote}}{}
\IfFileExists{microtype.sty}{% use microtype if available
  \usepackage[]{microtype}
  \UseMicrotypeSet[protrusion]{basicmath} % disable protrusion for tt fonts
}{}
\makeatletter
\@ifundefined{KOMAClassName}{% if non-KOMA class
  \IfFileExists{parskip.sty}{%
    \usepackage{parskip}
  }{% else
    \setlength{\parindent}{0pt}
    \setlength{\parskip}{6pt plus 2pt minus 1pt}}
}{% if KOMA class
  \KOMAoptions{parskip=half}}
\makeatother
\usepackage{xcolor}
\IfFileExists{xurl.sty}{\usepackage{xurl}}{} % add URL line breaks if available
\IfFileExists{bookmark.sty}{\usepackage{bookmark}}{\usepackage{hyperref}}
\hypersetup{
  pdftitle={Interactive Features of Loss Data Analytics},
  pdfauthor={An open text authored by the Actuarial Community},
  hidelinks,
  pdfcreator={LaTeX via pandoc}}
\urlstyle{same} % disable monospaced font for URLs
\usepackage{longtable,booktabs}
% Correct order of tables after \paragraph or \subparagraph
\usepackage{etoolbox}
\makeatletter
\patchcmd\longtable{\par}{\if@noskipsec\mbox{}\fi\par}{}{}
\makeatother
% Allow footnotes in longtable head/foot
\IfFileExists{footnotehyper.sty}{\usepackage{footnotehyper}}{\usepackage{footnote}}
\makesavenoteenv{longtable}
\usepackage{graphicx,grffile}
\makeatletter
\def\maxwidth{\ifdim\Gin@nat@width>\linewidth\linewidth\else\Gin@nat@width\fi}
\def\maxheight{\ifdim\Gin@nat@height>\textheight\textheight\else\Gin@nat@height\fi}
\makeatother
% Scale images if necessary, so that they will not overflow the page
% margins by default, and it is still possible to overwrite the defaults
% using explicit options in \includegraphics[width, height, ...]{}
\setkeys{Gin}{width=\maxwidth,height=\maxheight,keepaspectratio}
% Set default figure placement to htbp
\makeatletter
\def\fps@figure{htbp}
\makeatother
\setlength{\emergencystretch}{3em} % prevent overfull lines
\providecommand{\tightlist}{%
  \setlength{\itemsep}{0pt}\setlength{\parskip}{0pt}}
\setcounter{secnumdepth}{5}
\usepackage{booktabs}
\setcounter{secnumdepth}{2}
\usepackage[]{natbib}
\bibliographystyle{plainnat}

\title{Interactive Features of Loss Data Analytics}
\author{An open text authored by the Actuarial Community}
\date{}

\begin{document}
\maketitle

{
\setcounter{tocdepth}{2}
\tableofcontents
}
\hypertarget{preface}{%
\chapter*{Preface}\label{preface}}
\addcontentsline{toc}{chapter}{Preface}

\emph{Date: 22 September 2021}

This repository describes interactive features of the \emph{Loss Data Analytics} text that are being developed, in process, and in the planning stages. By \textbf{interactive features}, we start with simple things like layering content (such as Show/Hide way of displaying code below) and moving on from there.

R Code for Frequency Table

\hypertarget{display.T:Frequency.2Intro}{}
\begin{verbatim}
Insample <- read.csv("Insample.csv", header=T,  na.strings=c("."), stringsAsFactors=FALSE)
Insample2010 <- subset(Insample, Year==2010)
table(Insample2010$Freq)
\end{verbatim}

\hypertarget{introduction-to-interactive-features}{%
\chapter{Introduction to Interactive Features}\label{introduction-to-interactive-features}}

\hypertarget{active-learning}{%
\section{Active Learning}\label{active-learning}}

\href{https://cei.umn.edu/support-services/tutorials/what-active-learning}{Active Learning} is a phrase used by educators to mean an approach to classroom activities in which students engage the material they study through reading, writing, talking, listening, and reflecting. In contrast, a traditional instructor is sometimes referred to as the ``sage on the stage,'' where the teacher does most of the talking and students are passive.

Proponents of active learning suggest augmenting classroom activities by talking and listening in small groups, and having students write, read, and reflect on material. A variety of classroom tools have been forwarded for engaging students including discussion of scenarios and case studies, one minute papers, ``shared brainstorming,'' and the like. For more details, see a site sponsored by the University of Minnesota that provides \href{https://cei.umn.edu/active-learning}{a list of basic active learning activities} or a \href{https://cft.vanderbilt.edu/active-learning/}{summary} from the University of Vanderbilt.

Active learning is typically used in reference to classroom activities but it can also refer to textbooks. In our implementation of \emph{Loss Data Analytics}, we propose to include a number of \textbf{interactive} features that will allow a student to actively explore the content, described in Section \ref{S:Features}.

When taken outside of the classroom, the concept of active learning promotes one of the \href{http://www.hewlett.org/programs/education/deeper-learning}{deeper learning} goals set forth by some educational leaders: The ability to learn how to learn independently. We hope that instructors will be able to use the online text so that students can discover how to monitor and direct their own work and learning.

\hypertarget{S:Features}{%
\section{LDA Interactive Features}\label{S:Features}}

On the one hand, interactive features will help keep readers engaged as they explore the text. On the other hand, more is not necessarily better. Readers can be easily overwhelmed with a plethora of options and not appreciated the pedagogic approach with too many alternatives. So, we need to be thoughtful as we introduce interactive features.

\hypertarget{what-we-have}{%
\subsection{What we Have}\label{what-we-have}}

One easy feature that we have already adopted extensively is to layer the content through the use of \emph{Show/Hide} labels or text. Clicking on text that is bold, in different fonts, as well as color, allows viewers to reveal and to hide selected material. We use:

\begin{itemize}
\tightlist
\item
  Statistical (\texttt{R}) code available via \emph{Show/Hide} labels
\item
  Exercise Solutions available via \emph{Show/Hide} labels
\item
  Short demonstrations (``snippets'') of mathematics available via \emph{Show/Hide} labels
\end{itemize}

Another feature (not so easy to code) that we have introduced is quizzes that appear at the end of each section. These quizzes are low-level formative assessment tools; they allow viewers to check their comprehension of material that they have just read. More details on \textbf{End of Section Quizzes} appear in Section \ref{S:EndSectionQuizzes} of this summary.

We have also incorporated a number of \textbf{Animated Graphs} graphs in the simulation Chapter 6. \href{https://openacttexts.github.io/Loss-Data-Analytics/C-Simulation.html\#S:ImportanceSampling}{Here} is one demonstration. These are coded in \texttt{R} - although moving (animated), they do not require user interaction.

\hypertarget{in-process}{%
\subsection{In Process}\label{in-process}}

We are currently in the process of developing an \textbf{Online Glossary}. This is a feature that viewers regularly see in websites and so will not be overwhelming. The idea is to move the cursor/mouse over selected terms/phrases and have a definition appear. This is handy for insurance and statistical terms, as well as reminders of selected acronyms.

One of our big challenges will be to decide upon the type and extent of \textbf{Exercises} to include. See the Section \ref{S:Exercises} for a summary.

\hypertarget{plans-for-future-work}{%
\subsection{Plans for Future Work}\label{plans-for-future-work}}

There are a number of features that we now know how to implement in our environment (using \texttt{R} \textbf{Bookdown} package with Github) but have not yet initiated. In some cases, it is simply a matter of recruiting volunteers to author the content. In other cases, we need to convince ourselves that these additions will actually improve the work.

\begin{itemize}
\item
  \textbf{Video explanation of concepts}. This is easy to do with our system - see \href{https://ewfreesres.github.io/RegressModel/index.html}{the site} for an example. The challenge is to recruit people willing to author the content.
\item
  \textbf{Interactive statistical code}. As we move from a world of mathematical/probabilistic modeling to one that features data, we need to think more about how to integrate statistcal code into the pedagogy. See Section \ref{S:StatisticalCode} for a summary.
\item
  \textbf{Interactive graphs}. The \texttt{R} package \textbf{Shiny} allows us to introduce interactive graphs. See \href{https://ewfrees.github.io/LDARcode/index.html\#32_gamma_distribution}{the site} for an example. Part of the issue is getting volunteers to author the content. Another part is that this package requires a bit more overhead so it is not clear as to whether this gets integrated into the book or is placed in a supporting site.
\item
  \textbf{Links to external sources}. This is easy to do with our system. The challenge is getting the right set of references and setting up a system so that links can be maintained.
\end{itemize}

\hypertarget{S:NonFeatures}{%
\section{LDA Non-Interactive Features}\label{S:NonFeatures}}

Simply to round out the discussion, we point out features that are not interactive that still help to set the book project apart as special. These include:

\begin{itemize}
\tightlist
\item
  Collobarative, international author and review teams. There a variety of points of view represented in this work, making it stronger than if authored by an individual.
\item
  Offline versions of the text, via \emph{pdf} and \emph{EPUB}, are freely available.
\item
  Translations of the work will begin shortly, focusing on a Spanish version but with a Portugese one also in the works.
\end{itemize}

\#End of Section Quizzes \{\#S:EndSectionQuizzes\}

\hypertarget{introduction-to-loss-data-analytis}{%
\section{1. Introduction to Loss Data Analytis}\label{introduction-to-loss-data-analytis}}

\hypertarget{section-1.1}{%
\subsection{Section 1.1}\label{section-1.1}}

\hypertarget{surveyElement11}{}

\hypertarget{surveyResult11}{}

Show Quiz Solution

\hypertarget{display.Quiz11.2}{}
\begin{center}\rule{0.5\linewidth}{0.5pt}\end{center}

\hypertarget{section-1.2}{%
\subsection{Section 1.2}\label{section-1.2}}

\hypertarget{surveyElement12}{}

\hypertarget{surveyResult12}{}

Show Quiz Solution

\hypertarget{display.Quiz12.2}{}
\begin{center}\rule{0.5\linewidth}{0.5pt}\end{center}

\hypertarget{section-1.3}{%
\subsection{Section 1.3}\label{section-1.3}}

\hypertarget{surveyElement13}{}

\hypertarget{surveyResult13}{}

Show Quiz Solution

\hypertarget{display.Quiz13.2}{}
\begin{center}\rule{0.5\linewidth}{0.5pt}\end{center}

\hypertarget{S:Exercises}{%
\chapter{Exercises}\label{S:Exercises}}

\hypertarget{the-issue}{%
\section{The Issue}\label{the-issue}}

Users expect to see exercises in a textbook. For \emph{Loss Data Analytics}, we have many sources and types of exercises that we can draw upon. The issue is deciding upon the type to re-inforce the emphasis of the book.

Currently, the book features almost as many different exercise types as there are authors. Some chapters have no exercises, others have an extensive amount.

See also some questions in a \href{https://github.com/ewfrees/InteractiveLDA/tree/master/FutureWork}{Future Work} folder on this site.

These differing viewpoints are also reflected in the examples in the book. To illustrative, some chapters focus on mathmatical/statistical developments, some on data analytic aspects, some on business content. This is the good and bad associated with a heterogenous collection of authors.

We can reduce the effects of this heterogeneity by providing a large pool of exercises. Some may exist outside of the book. With a large pool of exercises, users (teachers) can draw upon the subset that reflects their area of interests.

\hypertarget{wordpress-exercises}{%
\section{Wordpress Exercises}\label{wordpress-exercises}}

Here is a series of exercises that guide the viewer through some of the theoretical foundations of \emph{Loss Data Analytics}. Each tutorial is based on one or more questions from the professional actuarial examinations -- typically the Society of Actuaries short term actuarial mathematics exam that used to be known as ``Exam C.'' These exercises are coded using \textbf{Wordpress}.

We can explore bringing them into the \texttt{R} \textbf{Bookdown} environment.

\hypertarget{frequency-modeling}{%
\subsection{2. Frequency Modeling}\label{frequency-modeling}}

\href{https://www.ssc.wisc.edu/~jfrees/loss-data-analytics/loss-data-analytics-problems/}{Frequency Distribution Guided Tutorials}

\hypertarget{modeling-loss-severity}{%
\subsection{3. Modeling Loss Severity}\label{modeling-loss-severity}}

\href{http://www.ssc.wisc.edu/~jfrees/loss-data-analytics/chapter-3-modeling-loss-severity/loss-data-analytics-severity-problems/}{Severity Distribution Guided Tutorials}

\hypertarget{model-selection-and-estimation}{%
\subsection{4. Model Selection and Estimation}\label{model-selection-and-estimation}}

\href{http://www.ssc.wisc.edu/~jfrees/loss-data-analytics/loss-data-analytics-model-selection/}{Model Selection Guided Tutorials}

\hypertarget{aggregate-loss-models}{%
\subsection{5. Aggregate Loss Models}\label{aggregate-loss-models}}

\href{https://www.ssc.wisc.edu/~jfrees/loss-data-analytics/aggregate-loss-guided-tutorials/}{Aggregate Loss Guided Tutorials}

\hypertarget{S:StatisticalCode}{%
\chapter{Statistical Code}\label{S:StatisticalCode}}

\begin{itemize}
\item
  As the baseline, we have already remarked that \texttt{R} statistical code appears in the text, although it does not overwhelm readers via the \emph{Show/Hide} features that we are using.
\item
  It will be straight forward to also include \textbf{Python} code in the same fashion.
\item
  We have a \href{https://ewfrees.github.io/LDARcode/index.html}{supporting site for statistical code} that allows viewers to get more experience analyzying loss data using \texttt{R}.
\item
  There are a variety of ways of learning coding procedures interactively. One option is \href{https://www.datacamp.com}{Datacamp}. In particular, they have developed an \texttt{R} package, \href{https://support.datacamp.com/hc/en-us/articles/360007749853-What-is-DataCamp-Light-}{Datacamp Light}. This allows us to integrate this interactive approach into our authoring system. \href{https://ewfreesres.github.io/RegressModel/index.html}{Here} is a link to an online regression tutorial done using this approach. There are a few little technical details but we think these are now ironed out. This is a promising option that will help make the book special.
\end{itemize}

\end{document}
